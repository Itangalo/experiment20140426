\part*{Problemuppgifter för matematikundervisning}

\chapter{Dura Schola Vitae, termin 6}

\setcounter{section}{0}
\section{Vilket betyg bör Alice få?}
Alice har läst matematik 1c. Vi vet följande om hennes kunskaper:

\begin{itemize}
  \item Alice hanterar exempelvis statistik, sannolikheter, förändringsfaktor och att sätta in värden i algebraiska uttryck (som inte omfattar potenser).
  \item Alice hanterar \textbf{inte} trigonometri, funktioner och algebraiska uttryck som omfattar parenteser.
  \item Alice kan visa alla föråmgor i kunskapskraven på E-nivå, så länge hon håller sig inom de moment som hon behärskar.
  \item På nationella provet skrev Alice ett E med lite marginal.
\end{itemize}

\noindent Nu är undervisningen slut och Alice ska få ett kursbetyg. Vilket betyg bör hon få? \emph{Motivera ditt svar.}



\newpage
\part*{Problemuppgifter för matematikundervisning}

\chapter{Dura Schola Vitae, termin 6}

\section{Bobs planering av matte 2c}
Bob tar över en klass som ska börja läsa matematik 2c. Han planerar att släppa boken och göra ett eget upplägg av kursen.

\subsection{Vilka moment ingår?}
Vilka av följande moment bör Bob ha med i sin planering av kursen?

\begin{itemize}
  \item Multiplicering av polynom
  \item Kvadratkomplettering
  \item Faktorisering av andragradsuttryck med hjälp av konjugat- och kvadreringsreglerna
\end{itemize}

\subsection{Vad har eleverna med sig?}
Vilka av följande saker kan Bob räkna med att eleverna har träffat på i tidigare undervisning? Vilka kan han räkna med att eleverna behärskar?

\begin{itemize}
  \item Räta linjens ekvation på formen ax + by + c = 0
  \item Riktningskoefficient som begrepp
  \item Metoder för att bestämma ekvationen för en rät linje
\end{itemize}

\noindent \emph{Motivera dina svar.}



\newpage
\part*{Problemuppgifter för matematikundervisning}

\chapter{Dura Schola Vitae, termin 6}

\section{Kunskapsläget för Eva}
Eva ska byta skola och få med sig ett omdöme. Ge ett exempel på hur hennes kunskapsläge i matematik kan sammanfattas. (Du får själv bestämma hur hennes kunskapsläge ser ut.)

\noindent \emph{Motivera det ramverk du väljer för att beskriva elevens kunskapsläge.}
