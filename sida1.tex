\part*{Problemuppgifter för matematikundervisning}

\chapter{Dura Schola Vitae}

\setcounter{section}{0}
\section{Vilket betyg bör Alice få?}
Alice har läst matematik 1c. Vi vet följande om hennes kunskaper:

\begin{itemize}
  \item Alice kan visa alla föråmgor i kunskapskraven, så länge hon håller sig inom de moment som hon behärskar.
  \item Moment som fungerar ok:
    \begin{itemize}
      \item promille, ppm, procentenheter förändringsfaktor
      \item statistik: läsa och tolka diagram och tabeller
      \item sannolikheter
      \item algebra: sätta in värden i algebraiska uttryck, grundläggande förenkling av algebraiska uttryck
      \item vektorer
    \end{itemize}
  \item Moment som Alice inte behärskar
    \begin{itemize}
      \item ekvationer, algebraiska uttryck som omfattar parenteser, olikheter
      \item implikation och ekvivalens
      \item trigonometri
      \item funktioner i allmänhet
    \end{itemize}
  \item På nationella provet skrev Alice ett E med lite marginal.
\end{itemize}

\noindent Nu är undervisningen slut och Alice ska få ett kursbetyg. Vilket betyg bör hon få? \em{Motivera ditt svar.}
