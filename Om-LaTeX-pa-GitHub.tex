\section{Om LaTeX på GitHub}

Några tips om du vill skriva LaTeX på GitHub:

\begin{itemize}

\item \textbf{Skriv varje mening på en ny rad.}
Det kan se lite konstigt ut i den råa textfilen, men det gör det möjligt för GitHub att hålla reda på ändringar (eftersom de loggas per rad).
\item \textbf{Lägg stora avsnitt i egna filer.}
Istället för att ha all text i ett jättedokument är det klokt att ha kapitel eller kanske avsnitt i egna filer.
Det gör det lättare att överblicka innehållet i projektet, och hitta de platser där man vill redigera.
Om du skapar en ny fil (med lägg till-knappen ovanför fillistan) kan du bädda in filens innehåll i huvuddokumentet genom LaTeX-kommandot \verb{\input{filens namn.tex}} eller \verb{\include{filens namn.tex}}.
Om du använder divis (-) istället för mellanslag slipper du krångla med specialtecken för att inkludera filerna i huvuddokumentet.
Samma sak gäller om du bara använder boktstäverna a--z och A--Z, och undviker specialtecken som å, ä och ö i filnamnet.

\end{itemize}

Det finns en rad andra tips man kan använda. Vi får lära oss efter hand.
