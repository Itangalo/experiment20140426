\section{Om GitHub}

Om du vill vara med och samarbera i det här projektet kommer du att behöva ett konto på GitHub.
GitHub är en öppen samarbetsplats främst för programmering, men webbplatsen kan också användas för andra typer av projekt.
Att skapa ett konto är gratis, och de allra flesta projekt på GitHub använder öppna licenser (GPL, Creative Commons och liknande).
GitHub är en bra plats, helt enkelt.


\subsection{Att utforska projekt}

Om du besöker ett projekt på GitHub kommer du att se en fillista för projektet, följt av innehållet i eventuell README-fil för projektet.
Du kan klicka på filer för att se innehållet i dem, och du kan använda länkstigen ovanför filinnehållet för att gå tillbaka till fillistningen.
Om du har ett konto på GitHub kan du också redigera filer som du tittar på.
När du gör det händer en av två saker:

\begin{itemize}

\item Om du är behörig att redigera i projektet kommer ändringarna att sparas direkt i filen när du är klar.
\item Om du inte är behörig att redigera i projektet kommer det att skapas en \emph{klon} av projektet, som du får redigera.
Klonen kan du ändra i som du vill, och du är nöjd med hur det blev kan du be ägaren av det ursprungliga projektet att dra in dina ändringar.
(Detta kallas på Git-språk att göra en \emph{merge}.)

\end{itemize}


\subsection{Ärendekön och wikin}

Varje projekt på GitHub har en ärendekö ("issue queue"), där man kan diskutera idéer och ändringar tillsammans innan de genomförs.
Det finns en länk till ärendekön i högerspalten.
Ärendekön borde vara självförklarande, men det kan vara värt att nämna att det går att prioritera ärenden olika och även tagga ärenden för att lättare hålla ordning på dem.

Varje projekt har dessutom en wiki. För kodningsprojekt är det ett bra sätt att sköta dokumentation.
Det här projektet är dock dokumentation i sig själv, och det är troligt att vi inte kommer att använda wikin.
