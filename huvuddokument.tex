\documentclass[12pt]{article}
\usepackage[utf8]{inputenc}
\usepackage{amsmath}
\usepackage{hyperref}
\usepackage{graphicx}
\usepackage[swedish]{babel}
\title{Experiment för samarbete på GitHub}
\date{}
\begin{document}
  \maketitle
  
  Du kan hitta projektet bakom det här dokumentet på \url{https://github.com/Itangalo/experiment20140426}.
  
  En pdf med senaste versionen av dokumentet hittar du på \url{http://tinyurl.com/20140426-compile}.
  
  \section{Introduktion}
  Det här dokumentet kommer från en fritt tillgänglig text på webbplatsen GitHub.
  Alla som är intresserade kan skapa sin egen variant av det här dokumentet, om man vill ändra på hur något ser ut.
  Vad som är ännu bättre: När man gjort ändringar kan man be att ändringarna (automatiskt) förs tillbaka in i det ursprungliga dokumentet.
  På GitHub finns också en ``issue queue'', där man kan föreslå och diskutera ändringar tillsammans.
  Ytterligare en fördel är att all historik för ändringar (inklusive vem som gjorde dem) sparas, så att man alltid kan följa utvecklingen, återställa vid behov, och se vilka som är med och bygger dokumentet.
  
  Flera personer kan göra ändringar i det ursprungliga dokumentet: för närvarande \href{https://github.com/Itangalo}{Johan Falk} och \href{https://github.com/ottve507}{Otto Velander}.
  (Vem som helst kan dock göra ändringar i sin egen version av dokumentet, som man själv blir ägare till och kan göra vad man vill med.)
  
  Innehållet i det här projektet är tillgänglig under Creative Commons-licens \href{http://creativecommons.org/licenses/by-nc-sa/3.0/}{attribution + non-commercial + share alike}.

  \section{LaTeX och inte}
  Just det här dokumentet är skrivet i LaTeX; ett markup-system som är vanligt för matematiska texter.
  Det är inte svårt att börja med (kolla hur källtexten ser ut och härma!), men det kan vara lite knepigare än en vanligt WYSIWYG-editor.
  Det finns dock ingenting som säger att man måste använda LaTeX på GitHub -- det går att lägga till alla sorters filer och versionshantera dem.

  \section{Om GitHub}

Om du vill vara med och samarbera i det här projektet kommer du att behöva ett konto på GitHub.
GitHub är en öppen samarbetsplats främst för programmering, men webbplatsen kan också användas för andra typer av projekt.
Att skapa ett konto är gratis, och de allra flesta projekt på GitHub använder öppna licenser (GPL, Creative Commons och liknande).
GitHub är en bra plats, helt enkelt.


\subsection{Att utforska och redigera projekt}

Om du besöker ett projekt på GitHub kommer du att se en fillista för projektet, följt av innehållet i eventuell README-fil för projektet.
Du kan klicka på filer för att se innehållet i dem, och du kan använda länkstigen ovanför filinnehållet för att gå tillbaka till fillistningen.
Om du har ett konto på GitHub kan du också redigera filer som du tittar på.
När du gör det händer en av två saker:

\begin{itemize}

\item Om du är behörig att redigera i projektet kommer ändringarna att sparas direkt i filen när du är klar.
\item Om du inte är behörig att redigera i projektet kommer det att skapas en \emph{klon} av projektet, som du får redigera.
Klonen kan du ändra i som du vill, och du är nöjd med hur det blev kan du be ägaren av det ursprungliga projektet att lägga till dina ändringar.
(Detta kallas på Git-språk att göra en \emph{merge}.)

\end{itemize}

När du gjort en redigering och ska spara ombeds du att skriva ett \emph{commit message}.
Det är en kort mening som sammanfattar dina ändringar (exempelvis ``La till ett stycke om vad commit message innebär'').
Håll gärna meningarna korta och tydliga.


\subsection{Ärendekön och wikin}

Varje projekt på GitHub har en ärendekö (``issue queue''), där man kan diskutera idéer och ändringar tillsammans innan de genomförs.
Det finns en länk till ärendekön i högerspalten.
Ärendekön borde vara självförklarande, men det kan vara värt att nämna att det går att prioritera ärenden olika och även tagga ärenden för att lättare hålla ordning på dem.

Varje projekt har dessutom en wiki. För kodningsprojekt är det ett bra sätt att sköta dokumentation.
Det här projektet är dock dokumentation i sig själv, och det är troligt att vi inte kommer att använda wikin.

Är tanken alltså att vi ska skriva ett dokument (som i och för sig kan bestå av inbäddade filer som den här)? Vad är det egentligen vi ska skriva om? jag har ju mina idéer som jag delvis redovisade på FB. Vilka idéer har ni andra? Nu skrev jag mest för att testa att redigera. Vi kan j uta bort den här texten sedan. 

  
  \section{Om LaTeX på GitHub}

Några tips om du vill skriva LaTeX på GitHub:

\begin{itemize}

\item \textbf{Skriv varje mening på en ny rad.}
Det kan se lite konstigt ut i den råa textfilen, men det gör det möjligt för GitHub att hålla reda på ändringar (eftersom de loggas per rad).
\item \textbf{Lägg stora avsnitt i egna filer.}
Istället för att ha all text i ett jättedokument är det klokt att ha kapitel eller kanske avsnitt i egna filer.
Det gör det lättare att överblicka innehållet i projektet, och hitta de platser där man vill redigera.
Om du skapar en ny fil (med lägg till-knappen ovanför fillistan) kan du bädda in filens innehåll i huvuddokumentet genom LaTeX-kommandot \verb+\input{filens-namn}+ eller \verb+\include{filens-namn}+.
Om du använder divis (-) istället för mellanslag slipper du krångla med specialtecken för att inkludera filerna i huvuddokumentet.
Samma sak gäller om du bara använder boktstäverna a--z och A--Z, och undviker specialtecken som å, ä och ö i filnamnet.

\end{itemize}

Det finns en rad andra tips man kan använda.
Vi får lära oss efter hand \todo{Wow! Det går att skriva kommentarer i marginalen genom kommandot \textbackslash todo\{kommentar\}. Lärde mig just på writelatex.com.}.


\end{document}
