\documentclass[12pt]{article}
\usepackage[utf8]{inputenc}
\usepackage{amsmath}
\usepackage{hyperref}
\usepackage{graphicx}
\userpackage[swedish]{babel}
\title{Experiment för samarbete på GitHub}
\date{}
\begin{document}
  \maketitle
  
  \section{Introduktion}
  Det här dokumentet kommer från en fritt tillgänglig text på webbplatsen GitHub.
  Alla som är intresserade kan skapa sin egen variant av det här dokumentet, om man vill ändra på hur något ser ut.
  Vad som är ännu bättre: När man gjort ändringar kan man be att ändringarna (automatiskt) förs tillbaka in i det ursprungliga dokumentet.
  På GitHub finns också en "issue queue", där man kan föreslå och diskutera ändringar tillsammans.
  Ytterligare en fördel är att all historik för ändringar (inklusive vem som gjorde dem) sparas, så att man alltid kan följa utvecklingen, återställa vid behov, och se vilka som är med och bygger dokumentet.
  
  Flera personer kan göra ändringar i det ursprungliga dokumentet: för närvarande Johan Falk och Otto Velander.
  (Vem som helst kan dock göra ändringar i sin egen version av dokumentet, som man själv blir ägare till och kan göra vad man vill med.)
  
  Innehållet i det här projektet är tillgänglig under Creative Commons-licens \href{http://creativecommons.org/licenses/by-nc-sa/3.0/}{attribution + non-commercial + share alike}.

  \section{LaTeX och inte}
  Just det här dokumentet är skrivet i LaTeX; ett markup-system som är vanligt för matematiska texter.
  Det är inte svårt att börja med (kolla hur källtexten ser ut och härma!), men det kan vara lite knepigare än en vanligt WYSIWYG-editor.
  Det finns dock ingenting som säger att man måste använda LaTeX på GitHub -- det går att lägga till alla sorters filer och versionshantera dem.

  \inlclude{Om_GitHub}
  
  \section{Om LaTeX på GitHub}

Några tips om du vill skriva LaTeX på GitHub:

\begin{itemize}

\item \textbd{Skriv varje mening på en ny rad.}
Det kan se lite konstigt ut i den råa textfilen, men det gör det möjligt för GitHub att hålla reda på ändringar (eftersom de loggas per rad).
\item \textbd{Lägg stora avsnitt i egna filer.}
Istället för att ha all text i ett jättedokument är det klokt att ha kapitel eller kanske avsnitt i egna filer.
Det gör det lättare att överblicka innehållet i projektet, och hitta de platser där man vill redigera.
Om du skapar en ny fil (med lägg till-knappen ovanför fillistan) kan du bädda in filens innehåll i huvuddokumentet genom LaTeX-kommandot \verbose{\include{filens namn.tex}}.

\end{itemize}

Det finns en rad andra tips man kan använda. Vi får lära oss efter hand.


\end{document}
